%==========================================================================
\chapter{\hypre{} Requirements}
\label{hypre Requirements}

%==========================================================================
\section{Functional Requirements}
\label{Functional Requirements}

\begin{description}

\item[R1] \hypre{} will consist of a set of linear solvers,
preconditioners, libraries of linear solvers, and an interface that
allows various applications to use these libraries and linear solvers
in various ways.
\begin{description}

\item[R1.1] Current linear solver include ILUT, AMG, EBE, PILUT, SMG,
and

\item[R1.2] Current libraries of linear solvers include PetsC, ISIS, and

\end{description}

\item[R2] \hypre{} must allow access to other libraries of linear
solvers directly through the \hypre{} interface and allow \hypre{}'s other
linear solvers and preconditioners to be accesses via the standard
interface for the other libraries of linear solvers.


\item[R3] New linear solvers, preconditioners, and libraries of
linear solvers should be able to be added to \hypre{} and accessed
through the \hypre{} interface any time without much modification.
\begin{description}

\item[R3.1] A degree of modularity and loose coupling should exist to
ensure modification and enhancements are simple.

\item[R3.2] (related to R2 also): every linear solver and
preconditioner should exist independently of the \hypre{} interface.

\end{description}

\item[R4] The applications that will use \hypre{} may provide data in a
variety of forms.  \hypre{} should allow new forms of data to be input
without much modification.
\begin{description}

\item[R4.1] Currently application data may consist of linear algebra,
finite elements, stencils, and

\item[R4.2] The \hypre{} interface should be modular to the extent that
adding new valid application data forms will be simple.

\end{description}

\item[R5] \hypre{} should provide mechanisms to maximize the number of
linear solvers and preconditioners avialable to an application
regardless of the format of that data provided by those applications.
\begin{description}

\item[R5.1] The user will choose which linear solver or preconditioner
to apply to the data. (A set of valid choices will be provided).

\item[R5.1] This should be accomplished primarily by "translating"
the application data into the storage schema required by the solver or
preconditioner or by applting the coefficient access method required
by the solver or preconditioner.

\end{description}

\end{description}

%==========================================================================
\section{Non-Functional Requirements}
\label{Non-Functional Requirements}

\begin{description}

\item[R6] \hypre{} should be portable across these platforms: Sun, DEC,
BLue, and
	
\item[R7] \hypre{} should work with applications written in C, C++,
FORTRAN77, and
	
\item[R8] \hypre{} should should be easy to compile, configure, and
install.

\item[R9] \hypre{} should be scalable.

\item[R10] \hypre{} should work with applications that utilize the MPI
standard as well as those incorporating both MPI and OpenMP or MPI and
Pthreads.

\end{description}

