\chapter{Interlanguage issues}

%==========================================================================
\section{Calling {\itshape hypre} from Fortran}

Fortran 77 subroutine arguments always pass copies of the argument addresses
on execution of the call.  This is referred to as call-by-address, or
call-by-reference.  In the called subroutine, the memory space at the
argument address can be altered.  The calling address, however, cannot be
altered.

C function parameters, on the other hand, are often directly copied upon
entry to the function.  These parameters do not have to be addresses.
This is referred to as call-by-value.  Altering the copied C parameter in the
called function has no effect on the parameter in the calling function.  So,
C essentially has more freedom than Fortran.  A pointer parameter in C can
achieve the same effect as call-by-reference, but all parameters do not have
to be pointers.

The simplicity of Fortran, and the flexibility of C, allow the languages to
interoperate.  However, portability might introduce complexities to
configuration fo the interoperation codes.

Portability across DEC Compass, DEC Tera, and IBM ASCI-Blue platforms is
achieved with the specific Fortran-calling-C mapping: 


  calling Fortran argument type        called C function parameter type
  -----------------------------        ------------------------------

   (addr of) integer*8           ---->            long int *
   (addr of) integer             ---->            int *
   (addr of) character           ---->            char *
   (addr of) double precision    ---->            double *


In particular, on DEC Compass, DEC Tera, and IBM ASCI-Blue, C-type
(long int *) points to a space that can hold an address, a space which
happens to be the size of that allocated by a Fortran integer*8 declaration.
In orther words, C can hold an address to anything in a (long int) and
assign that address to a Fortran integer*8 memory space (using (long int *)
call-by-value).

So, on the above platforms, Fortran can carry addresses in integer*8
variables.  These addresses might not have originated in any Fortran call,
and if they are addresses to inhomogeneously typed collectioins of data
(e.g. C structures), Fortran might not have enough information to dereference
the addresses.  But Fortran can still carry around such addresses, and in
particular hand them back and forth between various C functions.  The Hypre
Fortran interface makes extensive use of this technique.

Example:

  Fortran calling:

\begin{verbatim}

    integer*8        struct_addr
    integer          intg, ierr
    character        charact
    double precision double_prec

    call sub_name(struct_addr, intg, charact, double_prec, ierr)

\end{verbatim}


  C called:
  
\begin{verbatim}

    void hypre_C_NAME(sub_name)( long int  *struct_addr,
                                 int       *intg,
                                 char      *charact,
                                 double    *double_prec,
                                 int       *ierr          )
\end{verbatim}
%==========================================================================
