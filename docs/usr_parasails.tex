%==========================================================================
\section{ParaSails}

ParaSails is a parallel implementation of a sparse approximate inverse
preconditioner, using {\em a priori} sparsity patterns and least-squares
(Frobenius norm) minimization.  Symmetric positive definite (SPD) problems
are handled using a factored SPD sparse approximate inverse.  General
(nonsymmetric and/or indefinite) problems are handled with an
unfactored sparse approximate inverse.  It is also possible to
precondition nonsymmetric but definite matrices with a factored, SPD
preconditioner.

ParaSails uses {\em a priori} sparsity patterns that are patterns of powers
of sparsified matrices.  ParaSails also uses a post-filtering technique
to reduce the cost of applying the preconditioner.  
In advanced usage not described here, the pattern of the
preconditioner can also be reused to generate preconditioners for different
matrices in a sequence of linear solves.

For more details about the ParaSails algorithm, see \cite{Chow:1999:APS}.

\subsection{Synopsis}

\begin{display}
\begin{verbatim}
#include "HYPRE_parcsr_ls.h"

int HYPRE_ParCSRParaSailsCreate(MPI_Comm comm, HYPRE_Solver *solver, 
  int symmetry);
int HYPRE_ParCSRParaSailsSetParams(HYPRE_Solver solver, 
  double thresh, int nlevel, double filter);
int HYPRE_ParCSRParaSailsSetup(HYPRE_Solver solver, HYPRE_ParCSRMatrix A,
  HYPRE_ParVector b, HYPRE_ParVector x);
int HYPRE_ParCSRParaSailsSolve(HYPRE_Solver solver, HYPRE_ParCSRMatrix A,
  HYPRE_ParVector b, HYPRE_ParVector x);
int HYPRE_ParCSRParaSailsStats(HYPRE_Solver solver);
int HYPRE_ParCSRParaSailsDestroy(HYPRE_Solver solver);
\end{verbatim}
\end{display}

\subsection{Sparsity pattern}
\label{sparsity}
Mathematically, given a matrix $A$ and two parameters $m \ge 0$ and
{\em thresh}$\ge 0$,
the sparsity pattern for the preconditioner is the pattern of $\tilde{A}^m$
where $\tilde{A}$ is a binary matrix defined as
\[
\tilde{A}_{ij} = \left\{ \begin{array}{ll}
   1 & \mbox{if $i=j$ or $|(D^{-1/2} A D^{-1/2})_{ij}| >$ {\em thresh}} \\
   0 & \mbox{otherwise}
                 \end{array} \right.
\]
where the matrix
\[
D_{ii} = \left\{ \begin{array}{ll}
    |A_{ii}| & \mbox{if $|A_{ii}| > 0$} \\
    1 & \mbox{otherwise}
    \end{array} \right.
\]
and {\em thresh} is a nonnegative threshold.

\subsection{Minimization}
In a nonsymmetric matrix $A$, the approximate inverse $M$ is computed by
minimizing
\begin{equation}
\| I - M A \|_F
\label{eq:1}
\end{equation}
given a sparsity pattern for $M$.  If $A$ is symmetric, then a lower
triangular approximate inverse factor $G$ is computed by minimizing
\[
\| I - G L \|_F
\]
where $A=LL^T$ is the Cholesky factorization of $A$, and $G$ has
a lower triangularized sparsity pattern.

\subsection{Filtering}
After the values of $M$ have
been computed, small entries of $M$
may be dropped to reduce the cost of multiplying by $M$.
Entries in a matrix $\tilde{M}$ are dropped if they are smaller in
magnitude than a threshold, called the {\em filter value}.
The matrix $\tilde{M}$ is
scaled version of $M$ that makes the filter value independent of
the scaling of the original matrix $A$.  ($\tilde{M}$
is the approximate inverse that would have been computed if
the original $A$ was symmetrically diagonally scaled to have
a diagonal of all ones.)
In the SPD case, the factor $\tilde{M}$ is also rescaled.

\subsection{Interface functions}

A ParaSails solver {\tt solver} is returned with 
\begin{display}
\begin{verbatim}
int HYPRE_ParCSRParaSailsCreate(MPI_Comm comm, HYPRE_Solver *solver,
  int symmetry);
\end{verbatim}
\end{display}
where {\tt comm} is the MPI communicator.

The value of {\tt symmetry} has the following meanings, to indicate
the symmetry and definiteness of the problem, and to specify the 
type of preconditioner to construct:
\begin{center}
\begin{tabular}{|c|l|} \hline
value & meaning \\ \hline
0 & nonsymmetric and/or indefinite problem, and nonsymmetric preconditioner \\
1 & SPD problem, and SPD (factored) preconditioner \\
2 & nonsymmetric, definite problem, and SPD (factored) preconditioner \\ 
\hline
\end{tabular}
\end{center}
For more information about the final case, see section \ref{nearly}.

Parameters for setting up the preconditioner are specified using
\begin{display}
\begin{verbatim}
int HYPRE_ParCSRParaSailsSetParams(HYPRE_Solver solver, 
  double thresh, int nlevel, double filter);
\end{verbatim}
\end{display}

The parameters are used to specify the sparsity pattern and filtering value
(see above), and are described with suggested values as follows:

\begin{center}
\begin{tabular}{|c|c|c|c|c|l|} \hline
parameter    & type    & range                & sug. values  & default & meaning \\ \hline
{\tt nlevel} & integer & ${\tt nlevel} \ge 0$ & 0, 1, 2      & 1   & $m={\tt nlevel}+1$\\
\hline
{\tt thresh} & real    & ${\tt thresh} \ge 0$ & 0, 0.1, 0.01 & 0.1 & {\em thresh} $=$ {\tt thresh}\\
             &         & ${\tt thresh}  <  0$ & -0.75, -0.90 &     & {\em thresh} selected automatically\\
\hline
{\tt filter} & real    & ${\tt filter} \ge 0$ & 0, 0.05, 0.001 & 0.05 & filter value $=$ {\tt filter}\\
             &         & ${\tt filter}  <  0$ & -0.90        &     & filter value selected automatically\\
\hline
\end{tabular}
\end{center}

When ${\tt thresh} < 0$, then a threshold is selected such that 
$-{\tt thresh}$ represents the fraction of the nonzero elements
that are dropped.  For example, if ${\tt thresh} = -0.9$ then
$\tilde{A}$ will contain approximately ten percent of the nonzeros
in $A$.

When ${\tt filter} < 0$, then a filter value is selected such that 
$-{\tt filter}$ represents the fraction of the nonzero elements
that are dropped.  For example, if ${\tt filter} = -0.9$ then
approximately 90 percent of the entries in the computed approximate 
inverse are dropped.

The ParaSails preconditioner is constructed with
\begin{display}
\begin{verbatim}
int HYPRE_ParCSRParaSailsSetup(HYPRE_Solver solver, HYPRE_ParCSRMatrix A,
  HYPRE_ParVector b, HYPRE_ParVector x);
\end{verbatim}
\end{display}
where {\tt A} is a {\tt HYPRE\_ParCSRMatrix} that has been 
previously constructed.  The vectors {\tt b} and {\tt x} are
not used by this function.

A call to
\begin{display}
\begin{verbatim}
int HYPRE_ParCSRParaSailsSolve(HYPRE_Solver solver, HYPRE_ParCSRMatrix A,
  HYPRE_ParVector b, HYPRE_ParVector x);
\end{verbatim}
\end{display}
applies the ParaSails preconditioner to vector {\tt b}
and returns the result in {\tt x}.  The matrix {\tt A}
is not used by this function.

Statistics about the setup procedure can be printed to stdout using
\begin{display}
\begin{verbatim}
int HYPRE_ParCSRParaSailsStats(HYPRE_Solver solver);
\end{verbatim}
\end{display}
This is particularly useful to determine the actual values of 
{\tt thresh} and {\tt filter} used.

A call to
\begin{display}
\begin{verbatim}
int HYPRE_ParCSRParaSailsDestroy(HYPRE_Solver solver);
\end{verbatim}
\end{display}
deallocates the ParaSails solver.

\subsection{Choosing parameters}

The approximate inverse preconditioner is more accurate but more expensive
to construct and apply if there is a greater number of nonzeros in the
sparsity pattern, i.e., larger
values of {\tt nlevel} and smaller values of {\tt thresh}.  The default
parameters often produce a preconditioner that can be stored in less than
the space required to store the original matrix.
ParaSails does not need a large amount of intermediate storage in
order to construct the preconditioner.

\subsection{Preconditioning nearly symmetric matrices}
\label{nearly}
A nonsymmetric, but definite and nearly symmetric matrix $A$ 
may be preconditioned
with a symmetric preconditioner $M$.  Using a symmetric preconditioner
has a few advantages, such as guaranteeing positive
definiteness of the preconditioner, as well as being less expensive
to construct.

The nonsymmetric matrix $A$ must be definite,
i.e., $(A+A^T)/2$ is SPD, and the {\em a priori} sparsity pattern to be used
must be symmetric.  The latter may be guaranteed by 1) 
constructing the sparsity pattern with a symmetric matrix, or 2) if the
matrix is structurally symmetric (has symmetric pattern), then
thresholding to construct the pattern is not used (i.e.,
zero value of the {\tt thresh} parameter is used).

%==========================================================================
