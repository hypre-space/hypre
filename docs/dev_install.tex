%==========================================================================
\chapter{Installation}
\label{Installation}

There are three types of \hypre{} installations that are updated and
maintained by members of the Scalable Linear Solvers project:
\begin{enumerate}

\item The "alpha" installation (designated by an "a" in the version number):
   \begin{itemize}
   \item intended to give users access to relatively recent changes.
   \item updated somewhat frequently.
   \item may be relatively unstable.
   \item three old installations are retained.
   \item installed on:
      \begin{itemize}
      \item CASC Sun workstation cluster
      \item CASC Linux workstation cluster
      \end{itemize}
   \end{itemize}

\item The "beta" installation (designated by an "b" in the version number):
   \begin{itemize}
   \item intended to give users access to relatively recent changes.
   \item intended to give users consistent access across several platforms.
   \item may be relatively unstable.
   \item three old installations are retained.
   \item installed on:
      \begin{itemize}
       \item CASC Sun workstation cluster
       \item CASC Linux workstation cluster
       \item Compass Compaq cluster
       \item TeraCluster Compaq cluster
       \item Linux TeraCluster Compaq cluster
       \item Blue-Pacific
       \item Forest DEC cluster  (classified)
       \item Blue(Sky)-Pacific   (classified)
       \item White               (classified)
      \end{itemize}
   \end{itemize}

\item The general release installation:
   \begin{itemize}
   \item intended to give users consistent access across several platforms.
   \item updated less frequently than internal installation.
   \item should be relatively stable.
   \item three old installations are retained.
   \item installed on:
      \begin{itemize}
       \item CASC Sun workstation cluster
       \item CASC Linux workstation cluster
       \item Compass Compaq cluster
       \item TeraCluster Compaq cluster
       \item Linux TeraCluster Compaq cluster
       \item Blue-Pacific
       \item Forest DEC cluster  (classified)
       \item Blue(Sky)-Pacific   (classified)
       \item White               (classified)
      \end{itemize}
   \end{itemize}

\end{enumerate}

Two mailing list are available to inform users of the hypre
linear solver library when new releases become available.
Announcements about general releases made on the `hypre-announce'
mailing list, and announcements about beta releases are made on the
`hypre-beta-announce' list. 

Subscriptions to either mailing list is handled throught the LLNL
Majordomo list server, Majordomo@lists.llnl.gov.i To add yourself
to a mailing list, send mail to <Majordomo@list.llnl.gov> with
the following command in the body of the email message:

 subscribe hypre-announce

or

 subscribe hypre-beta-announce

%==========================================================================
\section{Installation Procedures}
\label{Installation Procedures}

The installation is broken down into 2 parts: 
  \begin{itemize}
  \item building the distribution (tar file)
  \item installation
  \end{itemize}

Build the \hypre{} distribution:

\begin{enumerate}

   \item Set the version number in \file{utilities/HYPRE\_utilities.h}.
   The version number, by convension, use the following syntax "M.mm.rr".
   Where the "M" is the major release number, "mm" is the minor
   relases number, and the "rr" an update number. Alpha, and Beta 
   releases are denoted by the following sytanx "M.nn.rra" for 
   alpha releases, and "M.nn.rrb" for beta releases. The version 
   number is used for the name of the release creation and is used
   in the tar file name and root directory for the distribution.

   \item Create a symbolic tag for the current version
   of the cvs repository. To create a generat release enter:
   \kbd{cvs rtag -b VM-nn-rr linear\_solvers}. Note that a
   general release requires that the cvs rtag branch option.
   Version numbers are mapped into a valid rtag branch name by the
   following conversion, "VM-nn-rr", note: the leading (capital)
   "V" and the version number separated with "-" replacing
   the "."s. This convension will over come the strict cvs
   tag name restrictions (e.g., begin with a letter, and no "."'s).
   The version information during the install will be derived
   from the main trunk archive using the information in the
   file utilites/HYPRE\_utilites.h. Alpha, and Beta releases
   are denoted by the following sytanx "VM-nn-rr-a" for alpha
   releases, and "VM-nn-rr-b" for beta releases. An alpha or
   beta release do not make branches in the repository. For a beta
   release the command would be similar to: 
   \kbd{cvs rtag VM-nn-rrb linear\_solvers}. Optionally, an rtag
   can be made relative to a date or time. For example set the
   rtag relative to last midnight enter:
   \kbd{cvs rtag -D 0:0 VM-nn-rrb linear\_solvers}.

   \item The distribution is created with the \file{mkdist} 
   Bourne-shell script located in the tools directory of the 
   \hypre{} repository.  Typing \kbd{mkdist -help} will give 
   general usage info. Enter: \kbd{mkdist VM-nn-rr}. This checks 
   out the rtag version (step 2 above) from the repository,
   reorganizes the file structure, build the documentation,
   and creates the tar file \file{hypre-M.nn.rr.tar.gz}.

\end{enumerate}

%==========================================================================
\section{Installation Example}
\label{Installation Example}

The following is an example session creating a beta distribution:
\begin{ttfamily}
\begin{mdseries}
\linebreak
\$ \textbf{pwd}\linebreak
/home/hypre\linebreak
\$ \textbf{cvs checkout linear\_solvers}\linebreak
U linear\_solvers/CHANGELOG\linebreak
U linear\_solvers/COPYRIGHT\_and\_DISCLAIMER\linebreak
 . . .\linebreak
U linear\_solvers/utilities/utilities.h\linebreak
U linear\_solvers/utilities/version\linebreak
\$ \textbf{cvs history -T}\linebreak
T 2000-09-18 23:56 +0000 treadway linear\_solvers [V1-0-0b1:A]\linebreak
T 2000-09-20 17:00 +0000 treadway linear\_solvers [V1-0-0b1:D]\linebreak
T 2000-09-20 20:08 +0000 treadway linear\_solvers [V1-0-0b1:D]\linebreak
T 2000-09-20 20:09 +0000 treadway linear\_solvers [V1-0-0b1:D]\linebreak
T 2000-10-12 20:27 +0000 treadway linear\_solvers [V1-3-1b:A]\linebreak
T 2001-01-11 20:58 +0000 treadway linear\_solvers [V1-4-0b:A]\linebreak
T 2001-01-11 22:07 +0000 treadway linear\_solvers [V1-4-0b:A]\linebreak
T 2001-01-11 22:12 +0000 treadway linear\_solvers [V1-4-0b:D]\linebreak
T 2001-01-11 22:15 +0000 treadway linear\_solvers [V1-4-0b:A]\linebreak
\$ \textbf{vi linear\_solvers/utilities/HYPRE\_utilities.h}\linebreak
 . . .\linebreak
\#define HYPRE\_Version() "HYPRE 1.4.0b \$Date$ Compiled: " \_\linebreak
\_DATE\_\_ " " \_\_TIME\_\_\linebreak
 . . .\linebreak
\#define HYPRE\_Version() "HYPRE 1.4.0b \$Date$ Compiled: " \_\linebreak
\_DATE\_\_ " " \_\_TIME\_\_\linebreak
"linear\_solvers/utilities/HYPRE\_utilities.h" 60 lines, 1570 characters\linebreak
\$ \textbf{cvs commit linear\_solvers/utilities/HYPRE\_utilities.h\linebreak
beta release 1.5.0b}\linebreak
CVS: ----------------------------------------------------------------------\linebreak
CVS: Enter Log.	 Lines beginning with `CVS:' are removed automatically\linebreak
CVS:\linebreak
CVS: Committing in linear\_solvers/utilities\linebreak
CVS:\linebreak
CVS: Modified Files:\linebreak
CVS:	HYPRE\_utilities.h\linebreak
CVS: ----------------------------------------------------------------------\linebreak
~\linebreak
""/tmp/cvsAAABga42v" 9 lines, 345 characters\linebreak
Checking in linear\_solvers/utilities/HYPRE\_utilities.h;\linebreak
/home/casc/repository/linear\_solvers/utilities/HYPRE\_utilities.h,v  <--  HYPRE\_utilities.h\linebreak
new revision: 2.2; previous revision: 2.1\linebreak
done\linebreak
\$ \textbf{linear\_solvers/utilities/version -number}\linebreak
1.5.0b\linebreak
\$ \textbf{cvs rtag V1-5-0b linear\_solvers}\linebreak
\$ \textbf{./mkdist V1-5-0b}\linebreak
U linear\_solvers/CHANGELOG\linebreak
U linear\_solvers/COPYRIGHT\_and\_DISCLAIMER\linebreak
 . . .\linebreak
U linear\_solvers/utilities/utilities.h\linebreak
U linear\_solvers/utilities/version\linebreak
checking the hostname... perrin\linebreak
checking the architecture... solaris\linebreak
 . . .\linebreak
creating hypre-1.5.0b.tar file ...\linebreak
\$ \textbf{ls -l hypre-1.5*}\linebreak
-rw-rw-r--   1 treadway treadway 1532225 Jul 11 15:17 hypre-1.5.0b.tar.gz\linebreak
\linebreak
hypre-1.5.0b:\linebreak
total 20\linebreak
-rw-rw----   1 treadway treadway    4384 Jul 11 15:17 CHANGELOG\linebreak
-rw-rw----   1 treadway treadway    1645 Jul 11 15:17 COPYRIGHT\_and\_DISCLAIMER\linebreak
drwxrwxr-x   2 treadway treadway     512 Jul 11 15:17 bin\linebreak
drwxrwxr-x   4 treadway treadway     512 Jul 11 15:17 docs\linebreak
drwxrwxr-x  27 treadway treadway    1024 Jul 11 15:17 src\linebreak
\$ \textbf{cp hypre-1.5.0b.tar.gz hypre/sparc-sun-solaris2.7/}\linebreak
\$ \textbf{cd hypre-1.5.0b/src/tools/mklibs hypre/sparc-sun-solaris2.7/}\linebreak
\$ \textbf{cd hypre/sparc-sun-solaris2.7/}\linebreak
\$ \textbf{ls}\linebreak
AUTOTEST             hypre-1.5.0b.tar.gz  mklibs\linebreak
\$ \textbf{./mklibs -b hypre-1.5.0b.tar.gz}\linebreak
checking the hostname... perrin\linebreak
checking the architecture... solaris\linebreak
 . . .\linebreak
Very-cleaning FEI\_mv ...\linebreak
Very-cleaning test ...\linebreak
\$ \textbf{ls}\linebreak
AUTOTEST             hypre-1.5.0b.tar.gz\linebreak
hypre-1.5.0b         mklibs\linebreak
\$ \textbf{ls hypre-1.5.0b}\linebreak
CHANGELOG                 debug                     lib\linebreak
COPYRIGHT\_and\_DISCLAIMER  docs                      src\linebreak
bin                       include\linebreak
\$ \textbf{ls hypre-1.5.0b/lib}\linebreak
libHYPRE\_DistributedMatrix.a\linebreak
libHYPRE\_DistributedMatrixPilutSolver.a\linebreak
libHYPRE\_Euclid.a\linebreak
libHYPRE\_FEI.a\linebreak
libHYPRE\_IJ\_mv.a\linebreak
libHYPRE\_LSI.a\linebreak
libHYPRE\_MatrixMatrix.a\linebreak
libHYPRE\_ParaSails.a\linebreak
libHYPRE\_blas.a\linebreak
libHYPRE\_parcsr\_ls.a\linebreak
libHYPRE\_parcsr\_mv.a\linebreak
libHYPRE\_seq\_mv.a\linebreak
libHYPRE\_sstruct\_ls.a\linebreak
libHYPRE\_sstruct\_mv.a\linebreak
libHYPRE\_struct\_ls.a\linebreak
libHYPRE\_struct\_mv.a\linebreak
libHYPRE\_superlu.a\linebreak
libHYPRE\_utilities.a\linebreak
libkrylov.a\linebreak
\$ \linebreak
\end{mdseries}
\end{ttfamily}
