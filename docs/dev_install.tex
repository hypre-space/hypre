%==========================================================================
\chapter{Installation}
\label{Installation}

The supported installation platforms (read pre-built \hypre{} releases)
is a dynamic, evolving list of LC/ASCI machines, reflecting a cross
section of the \hypre{} user community.  The list of platforms below are
current as of January 2005.
There are three types of \hypre{} installations that are updated and
maintained by members of the Scalable Linear Solvers project:
\begin{enumerate}

\item The "alpha" installation (designated by an "a" in the version number):
   \begin{itemize}
   \item - intended to give users access to relatively recent changes.
   \item - updated somewhat frequently.
   \item - may be relatively unstable.
   \item - three old installations are retained.
   \item - installed on:
      \begin{itemize}
      \item CASC Linux workstation cluster
      \end{itemize}
   \end{itemize}

\item The "beta" installation (designated by an "b" in the version number):
   \begin{itemize}
   \item - intended to give users access to relatively recent changes.
   \item - intended to give users consistent access across several platforms.
   \item - may be relatively unstable.
   \item - three old installations are retained.
   \item - installed on:
      \begin{itemize}
       \item CASC Linux workstation cluster
       \item Linux clusters - MCR, ALC, ILX, Pengra, Thunder
       \item IBM Platforms - Frost, UV
       \item Compaq cluster - GPS
       \item Classified Linux clusters - Adelie, Emperor, Lilac, ACE
       \item Classified IBM Platforms - White, UM
       \item Classified Compaq cluster - SC
      \end{itemize}
   \end{itemize}

\item The general release installation:
   \begin{itemize}
   \item - intended to give users consistent access across several platforms.
   \item - updated less frequently than internal installation.
   \item - should be relatively stable.
   \item - three old installations are retained.
   \item - installed on:
      \begin{itemize}
       \item CASC Linux workstation cluster
       \item Linux clusters - MCR, ALC, ILX, Pengra, Thunder
       \item IBM Platforms - Frost, UV
       \item Compaq cluster - GPS
       \item Classified Linux clusters - Adelie, Emperor, Lilac, ACE
       \item Classified IBM Platforms - White, UM
       \item Classified Compaq cluster - SC
      \end{itemize}
   \end{itemize}

\end{enumerate}

Two mailing list are available to inform users of the HYPRE
linear solver library when new releases become available.
Announcements about general releases made on the `hypre-announce'
mailing list, and announcements about beta releases are made on the
`hypre-beta-announce' list. 

Subscriptions to either mailing list is handled through the LLNL
Majordomo list server, Majordomo@lists.llnl.gov. To add yourself
to a mailing list, send mail to <Majordomo@list.llnl.gov> with
the following command in the body of the email message:

 subscribe hypre-announce

or

 subscribe hypre-beta-announce

%==========================================================================
\section{Installation Procedures}
\label{Installation Procedures}

The installation is broken down into 2 parts: 
  \begin{itemize}
  \item building the distribution tar file
  \item installing the tar file on appropriate platforms
  \end{itemize}

Build the \hypre{} distribution tar file:

\begin{enumerate}

   \item Set the version number in \file{configure.in}, which is in the
   \file{config} sub-directory, by editing the \kbd{m4\_define(HYPRE\_VERSION, 1.8.3b)}
   line, additionally the \kbd{m4\_define(HYPRE\_DATE, 2004/02/14)} and,
   \kbd{m4\_define(HYPRE\_TIME, 10:35:00)} lines can be updated as well.
   The version number, by convention, use the following syntax `M.mm.rr'.
   Where the `M' is the major release number, `mm' is the minor
   release number, and the `rr' an update number. Alpha releases are denoted 
   `M.nn.rra' and beta releases are `M.nn.rrb'. The version number is used 
   for the name of the release creation and is used in the tar file name and
   root directory for the distribution. 

   The \file{configure} file must be created and commited to the code repository by
   running the \kbd{config/boostrap} script followed by \kbd{cvs commit}.

   \item Create a symbolic tag for the current version of the CVS code 
   repository. The convention used for tag numbers is to prepend a `V' to the
   version number and change all `.' to `-'; thus overcoming the CVS strict 
   naming restrictions for tags.  Optionally, an rtag can be made relative to
   a date or time. 

   NOTE: A cvs rtag with branch option is required for general but not alpha or
   beta releases.

   Enter the following to create the needed tag:
   \kbd{cvs rtag -b VM-nn-rr linear\_solvers} -- for a general release. 
   \kbd{cvs rtag VM-nn-rrb linear\_solvers} -- for a beta release. 
   \kbd{cvs rtag VM-nn-rra linear\_solvers} -- for an alpha release. 

   To set the rtag relative to last midnight (for a beta release) enter:
   \kbd{cvs rtag -D 0:0 VM-nn-rrb linear\_solvers}.

   \item The distribution tar file is created with the \kbd{mkdist} Bourne-shell
   script located in the tools directory of the \hypre{} repository.  Typing 
   \kbd{mkdist -help} will give general usage information. 
   Enter: \kbd{mkdist VM-nn-rr}. This checks out the rtag version (step 2 above)
   from the repository, reorganizes the file structure, builds the documentation,
   and creates \file{hypre-M.nn.rr.tar.gz}.

\end{enumerate}

Installing the \hypre{} distribution tar file:

\begin{enumerate}

   \item Copy the tar file \file{hypre-M.nn.rr.tar.gz} to \file{/usr/gapps/hypre/}
   on the LC platforms and to \file{/usr/casc/hypre/} on the CASC platforms.
   Run the script \kbd{mkdistlinks} to create symbolic links between the tar 
   file and the appropriate systems and locations for the build. For the LC
   platforms, there are two install locations:
      \begin{itemize}
       \item \file{/usr/gapps/hypre/}\textit{canonical system name}
       \item \file{/usr/casc/hypre/}\textit{canonical system name}
      \end{itemize}

   The \textit{canonical system name} is determined by \kbd{config.guess}. 
   The appropriate directory is found using the command:
      \begin{itemize}
       \item \file{/usr/gapps/hypre/`/usr/gapps/hypre/config.guess`}
       \item \file{/usr/casc/hypre/`/usr/gapps/hypre/config.guess`}
      \end{itemize}

   However the name returned by \kbd{config.guess} may be insufficient to
   distinguish between similar systems (e.g. the LINUX systems all return
   i686-pc-linux-gnu but each has differning systems). In this case the name of
   the platform (i.e. mcr or ilx) may be appended to the returned name.
   
   The user's of \hypre{} needn't deal with these details for the non-CASC systems,
   because the \file{/usr/apps/hypre} directory will automatically be linked to the
   appropate hypre system installation through a \kbd{SYS\_TYPE} environment 
   variable.

   Some mappings between the \file{/usr/gapps/hypre/} \textit{canonical system name}
   directory names, the \kbd{SYS\_TYPE} and the system are:
\begin{enumerate}
   \item \textit{canonical system name}  \kbd{SYS\_TYPE}       \kbd{System}
   \begin{itemize}
   \item  \file{alphaev68-dec-osf5.1}        tru64\_5               gps
   \item  \file{i686-pc-linux-gnu-ilx}       chaos\_2\_ia32         ilx
   \item  \file{i686-pc-linux-gnu-pengra}    chaos\_2\_ia32\_elan3  mcr, pengra
   \item  \file{ia64-unknown-linux}          chaos\_2\_ia32\_elan4  thunder
   \item  \file{powerpcll-ibm-aix5.1.0.0}    aix\_5\_ll             frost
   \end{itemize}
\end{enumerate}
   see: 
\htmladdnormallink{https://lc.llnl.gov/computing/techbulletins/bulletin258l.html}
{https://lc.llnl.gov/computing/techbulletins/bulletin258l.html}

   In order to copy the tar file from the open LC platforms to the classified 
   ones, the person doing the transfer must have an active account on the File
   Interchange System (FIS) see: 
\htmladdnormallink{http://www-lc.llnl.gov:6336/dynaweb/LCdocs/fis/}
{http://www-lc.llnl.gov:6336/dynaweb/LCdocs/fis/}

   \item Run the Bourne-shell install script \kbd{mklibs}, which is located in 
   the tools directory of the \hypre{} repository or the top-level installation 
   directory, to untar the distribution and build the \hypre{} libraries. 
   There are a couple specialized versions of \kbd{mklibs}, including:
   \kbd{mklibs.aix}, which creates combined 32, and 64 bit libraries
   \kbd{mklibs.babel} which adds --with-babel to each of the installations. 

   Typing \kbd{mklibs -help} will provide usage information. 
  
   Untarring the distribution tar file will create a directory named \file{hypre-M.nn.rr}
   which contains the following subdirectories:
      \begin{itemize}
       \item \file{bin}     contains hypre utilities
       \item \file{docs}    PostScript, PDF, and HTML documentation
       \item \file{src}     source code
      \end{itemize}

   The build process creates these directories:
      \begin{itemize}
       \item \file{debug}   compiled with debugging information
       \item \file{include} include files
       \item \file{lib}     optimized parallel compiled libraries
       \item \file{serial}  serially compiled libraries
       \item \file{share}   the Babel SIDL file
       \item \file{threads} OpenMP compiled libraries
      \end{itemize}

   At a minimum an optimized and a debugged version of the library will be 
   generated. Optionally, OpenMP versions of the libraries will be built and
   installed in a \file{threads} and \file{threads/debug} directory, assuming
   the target system supports OpenMP. The users will see this located at: 
   \file{/usr/apps/hypre/hypre-M.nn.rr}.

   \kbd{mklibs} script has an optional third argument \kbd{install} which will
   symbolically link the distribution to the `common' user accessible directories,
   depending on the installion type specified. Typical command to build a general
   distribution is:\linebreak
   \kbd{mklibs -g hypre-M.nn.rr.tar.gz install}\linebreak

   This would create the following links:\linebreak
   \file{/usr/apps/hypre/bin     -> /usr/gapps/hypre/.../hypre-M.nn.rr/bin}\linebreak
   \file{/usr/apps/hypre/debug   -> /usr/gapps/hypre/.../hypre-M.nn.rr/debug}\linebreak
   \file{/usr/apps/hypre/docs    -> /usr/gapps/hypre/.../hypre-M.nn.rr/docs}\linebreak
   \file{/usr/apps/hypre/include -> /usr/gapps/hypre/.../hypre-M.nn.rr/include}\linebreak
   \file{/usr/apps/hypre/lib     -> /usr/gapps/hypre/.../hypre-M.nn.rr/lib}\linebreak
   \file{/usr/apps/hypre/src     -> /usr/gapps/hypre/.../hypre-M.nn.rr/src}\linebreak
   \file{/usr/apps/hypre/threads -> /usr/gapps/hypre/.../hypre-M.nn.rr/threads}\linebreak
   \file{/usr/apps/hypre/alpha   -> /usr/gapps/hypre/.../hypre-M.nn.rr}\linebreak
   \file{/usr/apps/hypre/beta    -> /usr/gapps/hypre/.../hypre-M.nn.rr}\linebreak

   A beta install command of:\linebreak
   \kbd{mklibs -b hypre-M.nn.rrb.tar.gz install}\linebreak

   Would create the following links:\linebreak
   \file{/usr/apps/hypre/alpha -> /usr/gapps/hypre/.../hypre-M.nn.rrb}\linebreak
   \file{/usr/apps/hypre/beta  -> /usr/gapps/hypre/.../hypre-M.nn.rrb}\linebreak

   An alpha install command of:\linebreak
   \kbd{mklibs -a hypre-M.nn.rra.tar.gz install}\linebreak

   Would create the following links:\linebreak
   \file{/usr/apps/hypre/alpha -> /usr/gapps/hypre/.../hypre-M.nn.rra}\linebreak
   NOTE: in general, alpha releases are only done on CASC systems.

\end{enumerate}

%==========================================================================
\section{Installation Example}
\label{Installation Example}

The following example show the creation of a beta distribution being executed on 
a CASC System (tux cluster).

\begin{ttfamily}
\begin{mdseries}
\linebreak
\$ \textbf{pwd}\linebreak
/home/hypre\linebreak
\$ \textbf{cvs checkout linear\_solvers}\linebreak
\begin{verbatim}
U linear_solvers/CHANGELOG
U linear_solvers/COPYRIGHT_and_DISCLAIMER
 . . .
U linear_solvers/utilities/utilities.h
U linear_solvers/utilities/version
\end{verbatim}
\$ \textbf{cd linear\_solvers/config}\linebreak
\$ \textbf{vi configure.in}\linebreak
\begin{verbatim}
 . . .
m4_define(HYPRE_VERSION, 1.7.0a)
m4_define(HYPRE_DATE, 2001/08/24)
m4_define(HYPRE_TIME, 20:26:53)
 . . .
\end{verbatim}
\begin{bfseries}
\begin{verbatim}
m4_define(HYPRE_VERSION, 1.8.3b)
m4_define(HYPRE_DATE, 2003/11/13)
m4_define(HYPRE_TIME, 07:25:23)
\end{verbatim}
\end{bfseries}
"configure.in" 423 lines, 13216 characters\linebreak
\$ \textbf{cd ..}
\$ \textbf{./config/bootstrap}
\$ \textbf{cvs commit config/configure.in configure}\linebreak
\textbf{beta release 1.8.3b}\linebreak
\begin{verbatim}
Release 1.8.3b
CVS: ----------------------------------------------------------------------
CVS: Enter Log.  Lines beginning with `CVS:' are removed automatically
CVS:
CVS: Modified Files:
CVS:   configure.ac configure
CVS: ----------------------------------------------------------------------
~
""/tmp/cvsAAABga42v" 9 lines, 345 characters
Checking in linear_solvers/configure.ac;
/home/casc/repository/linear_solvers/configure.ac,v  <--  configure.ac
new revision: 2.34; previous revision: 2.33
done
Checking in configure;
/home/casc/repository/linear_solvers/configure,v  <--  configure
new revision: 2.107; previous revision: 2.106
done
\end{verbatim}
:q
\$ \textbf{cvs rtag V1-8-3b linear\_solvers}\linebreak
\$ \textbf{./mkdist V1-8-3b}\linebreak
\begin{verbatim}
U linear_solvers/CHANGELOG
U linear_solvers/COPYRIGHT_and_DISCLAIMER
 . . .
U linear_solvers/utilities/utilities.h
U linear_solvers/utilities/version
checking the hostname... tux149
checking the architecture... linux
 . . .
creating hypre-1.8.3b.tar file ...
\end{verbatim}
\$ \textbf{ls -l hypre-1.8*}\linebreak
\begin{verbatim}
-rw-rw-r--   1 hill66 hill66 1532225 Nov 13 07:32 hypre-1.8.3b.tar.gz

hypre-1.7.0b:
total 20
-rw-rw----   1 hill66 hill66    4384 Nov 13 07:32 CHANGELOG
-rw-rw----   1 hill66 hill66    1645 Nov 13 07:32 COPYRIGHT_and_DISCLAIMER
drwxrwxr-x   2 hill66 hill66     512 Nov 13 07:32 bin
drwxrwxr-x   4 hill66 hill66     512 Nov 13 07:32 docs
drwxrwxr-x  27 hill66 hill66    1024 Nov 13 07:32 src
\end{verbatim}
\$ \textbf{hypre-1.8.3b/src/utilities/version -number}\linebreak
1.8.3b\linebreak
\$ \textbf{scp hypre-1.8.3b.tar.gz frost:/usr/gapps/hypre}\linebreak
\$ \textbf{scp hypre-1.8.3b.tar.gz frost:/usr/casc/hypre}\linebreak
\$ \textbf{ftp fis}\linebreak
\begin{verbatim}
Connected to fis.llnl.gov.
220-                       NOTICE TO USERS
220-This is a Federal computer system and is the property of the
 . . .
220 reebok.llnl.gov FTP server (Version LLNL-22 built 08/13/01 07:32:54) ready.
Name (fis:hill66):
331 Password required for hill66.
Password:
230 User hill66 logged in.
\end{verbatim}
ftp> \textbf{cd TO}\linebreak
250 CWD command successful.\linebreak
ftp> \textbf{binary}\linebreak
200 Type set to I.\linebreak
ftp> \textbf{put hypre-1.8.3b.tar.gz}\linebreak
200 PORT command successful.\linebreak
150 Opening BINARY mode data connection for hypre-1.8.3b.tar.gz.\linebreak
226 Transfer complete.\linebreak
local: hypre-1.8.3b.tar.gz remote: hypre-1.8.3b.tar.gz\linebreak
1663281 bytes sent in 0.49 seconds (3342.89 Kbytes/s)\linebreak
ftp> \textbf{quit}\linebreak
221 Goodbye.\linebreak
\$ \textbf{ssh frost}\linebreak
 . . .\linebreak
\$ \textbf{cd /usr/gapps/hypre}\linebreak
\$ \textbf{ls}\linebreak
\begin{verbatim}
aix_4                        hypre-1.8.3a.tar.gz
aix_4ll                      i686-pc-linux-gnu
aix_5                        i686-pc-linux-gnu-chaos
aix_5_ll                     i686-pc-linux-gnu-pengra
alphaev56-dec-osf4.0f        i686-pc-linux-gnu-vivid
alphaev56-dec-osf5.1         irix64
alphaev67-dec-osf5.0         irix_6.5_64
alphaev67-dec-osf5.1         mips-sgi-irix6.5
alphaev67-unknown-linux-gnu  mkdirlinks
alphaev68-dec-osf5.1         mkdistlinks
chaos_2_ia32                 mklibs
config.guess                 mklibs.aix
config.sub                   mklibs.babel
env.blue                     mklibs.tc2k
env.linux                    powerpc-ibm-aix4.3.3.0
hypre-1.2.0.tar.gz           powerpc-ibm-aix5.1.0.0
hypre-1.3.1b.tar.gz          powerpcll-ibm-aix4.3.3.0
hypre-1.4.0b.tar.gz          powerpcll-ibm-aix5.1.0.0
hypre-1.5.0b.tar.gz          redhat_7_ia32
hypre-1.6.0.tar.gz           tru64_5sc
\end{verbatim}
\$ \textbf{./mkdistlinks hypre-1.8.3b.tar.gz}\linebreak
\begin{verbatim}
lrwxrwxrwx   1 hill66 hypre  22 Nov 13 07:32 alphaev67-dec-osf5.1/hypre-1.8.3b.tar.gz -> ../hypre-1.8.3b.tar.gz
lrwxrwxrwx   1 hill66 hypre  22 Nov 13 07:32 alphaev67-unknown-linux-gnu/hypre-1.8.3b.tar.gz -> ../hypre-1.8.3b.tar.gz
lrwxrwxrwx   1 hill66 hypre  22 Nov 13 07:32 alphaev68-dec-osf5.1/hypre-1.8.3b.tar.gz -> ../hypre-1.8.3b.tar.gz
lrwxrwxrwx   1 hill66 hypre  22 Nov 13 07:32 i686-pc-linux-gnu/hypre-1.8.3b.tar.gz -> ../hypre-1.8.3b.tar.gz
lrwxrwxrwx   1 hill66 hypre  22 Nov 13 07:32 i686-pc-linux-gnu-pengra/hypre-1.8.3b.tar.gz -> ../hypre-1.8.3b.tar.gz
lrwxrwxrwx   1 hill66 hypre  22 Nov 13 07:32 mips-sgi-irix6.5/hypre-1.8.3b.tar.gz -> ../hypre-1.8.3b.tar.gz
lrwxrwxrwx   1 hill66 hypre  22 Nov 13 07:32 powerpc-ibm-aix5.1.0.0/hypre-1.8.3b.tar.gz -> ../hypre-1.8.3b.tar.gz
lrwxrwxrwx   1 hill66 hypre  22 Nov 13 07:32 powerpcll-ibm-aix5.1.0.0/hypre-1.8.3b.tar.gz -> ../hypre-1.8.3b.tar.gz
\end{verbatim}
\$ \textbf{cd `/usr/gapps/hypre/config.guess`}\linebreak
\$ \textbf{ls}\linebreak
\begin{verbatim}
AUTOTEST             hypre-1.2.0.tar.gz   hypre-1.8.3a
STLport-4.0          hypre-1.3.1b         hypre-1.8.3a.tar.gz
alpha                hypre-1.4.0b         hypre-1.8.3b.tar.gz
beta                 hypre-1.4.0b.tar.gz  include
bin                  hypre-1.5.0b         lib
debug                hypre-1.5.0b.tar.gz  src
docs                 hypre-1.6.0          threads
hypre-1.2.0          hypre-1.6.0.tar.gz
\end{verbatim}
\$ \textbf{../mklibs -b hypre-1.8.3b.tar.gz}\linebreak
\begin{verbatim}
checking the hostname... frost
checking the architecture... aix
 . . .
Very-cleaning FEI_mv ...
Very-cleaning test ...
\end{verbatim}
\$ \textbf{ls}\linebreak
\begin{verbatim}
AUTOTEST             hypre-1.2.0.tar.gz   hypre-1.8.3a
STLport-4.0          hypre-1.3.1b         hypre-1.8.3a.tar.gz
alpha                hypre-1.4.0b         hypre-1.8.3b
beta                 hypre-1.4.0b.tar.gz  hypre-1.8.3b.tar.gz
bin                  hypre-1.5.0b         include
debug                hypre-1.5.0b.tar.gz  lib
docs                 hypre-1.6.0          src
hypre-1.2.0          hypre-1.6.0.tar.gz   threads
\end{verbatim}
\$ \textbf{ls hypre-1.8.3b}\linebreak
\begin{verbatim}
CHANGELOG                 debug                     src
COPYRIGHT_and_DISCLAIMER  docs                      threads
README                    include
bin                       lib
\end{verbatim}
\$ \textbf{ls beta/lib}\linebreak
libHYPRE\_DistributedMatrix.a\linebreak
libHYPRE\_DistributedMatrixPilutSolver.a\linebreak
libHYPRE\_Euclid.a\linebreak
libHYPRE\_FEI.a\linebreak
libHYPRE\_IJ\_mv.a\linebreak
libHYPRE\_LSI.a\linebreak
libHYPRE\_MatrixMatrix.a\linebreak
libHYPRE\_ParaSails.a\linebreak
libHYPRE\_parcsr\_ls.a\linebreak
libHYPRE\_parcsr\_mv.a\linebreak
libHYPRE\_seq\_mv.a\linebreak
libHYPRE\_sstruct\_ls.a\linebreak
libHYPRE\_sstruct\_mv.a\linebreak
libHYPRE\_struct\_ls.a\linebreak
libHYPRE\_struct\_mv.a\linebreak
libHYPRE\_superlu.a\linebreak
libHYPRE\_utilities.a\linebreak
libHYPRE\_krylov.a\linebreak
\$ \linebreak
\end{mdseries}
\end{ttfamily}

Variations of mklib and use include:
\begin{bfseries}
\begin{verbatim}
Name          Purpose
mklibs        does default, debug, serial, and opt. threaded
mklibs.aix    build 32 and 64 bit libraries on white & frost
mklibs.babel  does the default plus builds babel libraries

Name          Supported host
mklibs        gps, sc
mklibs.aix    frost and white
mklibs.babel  Solaris, Linux
\end{verbatim}
\end{bfseries}
