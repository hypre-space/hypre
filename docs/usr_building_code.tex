%==========================================================================
\chapter{Building the Code}
\section{Code configuration}

To automatically generate machine specific makefiles, type
\kbd{configure} in the top level directory.  The configure
script is a portable script generated by GNU Autoconf.  It runs a
series of tests to determine characteristics of the machine on which
it is running, and it uses the results of the these tests to produce
the machine specific makefiles, called `Makefile', from template files
called `Makefile.in' in each directory.  Once the makefiles are
produced you can run make as you would with any other makefile.

The configure script primarily does the following things:
\begin{itemize}
\item selects a compiler
\item provides either optimization or debugging options for the compiler
\item finds the headers and libraries for MPI
\end{itemize}

The configure script makes these decisions based on a hierarchical check.  First, it attempts to identify the machine on which it is running as a specific supported machine.  Next it will try to identify the architecture as a supported architecture.  If both of these fail, generic default decisions are made by the script.  However, the script does have some command-line options that can give you control over the choices it will make.  You can type
\verb+configure --help+
to see the list of all of the command-line options to configure. This is
the best resource for information on configure options.  Below is some
additional helpful information.  Be aware that not all command line options
have been tested on all machines and architectures, even supported machines and architectures. 


\begin{description}

\item[--with- options] Each --with- option that is listed in configure--help also was a --without- option (usually the default, but not always).  Additionally, all --with- options have a --with-+{\itshape option}=+{\itshape pathname}.  This is not a supported feature for all --with-options however and may have no effect on configuration.  

\item[Compilers] If you want to choose a compiler then is it recommended
that you choose all (C, C++, Fortran) compilers.

\item[Compiler Flags] To choose optimization or debug, use
\verb+--enable-opt+ (default) or \verb+--enable-debug+.
For other compiler flags use the \verb+--with-CFLAGS+ option.  

\item[BLAS] The optimum BLAS for the platform should be obtained
without specification of a configure option.  In other words, by default, the systems native optimized BLAS library will the automatically chosen.
To specify another BLAS library available on a
platform, use \verb+--with-BLAS=+{\itshape pathname} or
\verb+--with-BLAS=+{\itshape link list}.  To configure and compile without BLAS use the --without-BLAS option.  
\end{description}

Configure automatically generates a file \verb+HYPRE_config.h+ that
includes all the header files found to be necessary by configure.
This file may be used to see how a compiled version of the library
was configured and may also be included by the user in his/her own code.

\section{Linking into another program}

A program linking with {\slshape hypre} must be compiled with
\verb+-I$HYPRE_DIR/include+ and linked with
\verb+-L$HYPRE_DIR/lib -l+{\itshape hypre library name}... 
\verb+-l+{\itshape hypre library name}.
When using a formal {\slshape hypre} installation, \verb+$HYPRE_DIR+ is
\verb+/usr/apps/hypre+.
When using a private installation, the installer sets \verb+$HYPRE_DIR+.
Additionally, any other libraries to which {\slshape hypre} is linked must also be linked to by the users application e.g. the BLAS library or PETSc library are often (but not always) linked in by {\slshape hypre} and must therefore be linked in by the users application as well.  It may be useful to reference the Makefile in the test subdirectory.  This Makefile is designed to build a test application that links with and calls {\slshape hypre}.  All include and linking flags that are used by {\slshape hypre} and needed by the application get exported to this file by the configure script.  
