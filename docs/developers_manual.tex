\documentclass{article}
\usepackage{docxx}
\begin{document}
\pagebreak
\sloppy
\begin{cxxContents}
\cxxContentsEntry{1}{Hypre Developer's Manual}{}
\begin{cxxContents}
\cxxContentsEntry{1.1}{Hypre Requirements}{}
\begin{cxxContents}
\cxxContentsEntry{1.1.1}{Functional Requirements}{}
\cxxContentsEntry{1.1.2}{Non-Functional Requirements}{}
\end{cxxContents}
\cxxContentsEntry{1.2}{Hypre Design}{}
\begin{cxxContents}
\end{cxxContents}
\cxxContentsEntry{1.3}{Makefile Standards}{}
\cxxContentsEntry{1.4}{Documentation Standards}{}
\cxxContentsEntry{1.5}{Coding Standards}{}
\end{cxxContents}
\cxxContentsEntry{2}{Quaility Assurance Issues}{}
\cxxContentsEntry{3}{Installation Procedures}{}
\cxxContentsEntry{4}{Code Reference}{}
\end{cxxContents}
\begin{cxxentry}
{}
        {Hypre Developer's Manual}
        {}
        {}
        {1}
\cxxVersion{
1.0
\begin{center}
Linear Solvers Group
\end{center}
\begin{center}
Center for Applied Scientific Computing
\end{center}
\begin{center}
Lawrence Livermore National Laboratory
\end{center}
\begin{center}
\today
\end{center}
\strut}
\begin{cxxdoc}

\end{cxxdoc}
\begin{cxxnames}
\cxxitem{}
        {Hypre Requirements}
        {}
        {}
        {1.1}
\cxxitem{}
        {Hypre Design}
        {}
        {}
        {1.2}
\cxxitem{}
        {Error Handling}
        {}
        {}
        {}
\label{cxx.1.6}
\cxxitem{}
        {Makefile Standards}
        {}
        {}
        {1.3}
\cxxitem{}
        {Documentation Standards}
        {}
        {}
        {1.4}
\cxxitem{}
        {Coding Standards}
        {(...)}
        {}
        {1.5}
\end{cxxnames}
\begin{cxxentry}
{}
        {Hypre Requirements}
        {}
        {}
        {1.1}
\begin{cxxnames}
\cxxitem{}
        {Functional Requirements}
        {}
        {}
        {1.1.1}
\cxxitem{}
        {Non-Functional Requirements}
        {}
        {}
        {1.1.2}
\end{cxxnames}
\begin{cxxentry}
{}
        {Functional Requirements}
        {}
        {}
        {1.1.1}
\begin{cxxdoc}

R1.	Hypre will consist of a set of linear solvers, preconditioners,
libraries of linear solvers, and an interface that allows various 
applications to use these libraries and linear solvers in various ways.

R1.1: Current linear solver include ILUT, AMG, EBE, PILUT, SMG, and


R1.2: Current libraries of linear solvers include PetsC, ISIS, and



R2.	Hypre must allow access to other libraries of linear solvers
directly through the hypre interface and allow hypre's other linear
solvers and preconditioners to be accesses via the standard interface 
for the other libraries of linear solvers.


R3.	New linear solvers, preconditioners, and libraries of linear solvers 
should be able to be added to hypre and accessed through the hypre
interface any time without much modification.

R3.1: A degree of modularity and loose coupling should exist to 
ensure modification and enhancements are simple.

R3.2 (related to R2 also): every linear solver and preconditioner
should exist independently of the hypre interface.


R4.	The applications that will use hypre may provide data in a variety
of forms.  Hyper should allow new forms of data to be input
without much modification.

R4.1: Currently application data may consist of linear algebra, 
finite elements, stencils, and

R4.2: The hypre interface should be modular to the extent that adding
new valid application data forms will be simple.


R5.	Hypre should provide mechanisms to maximize the number of linear
solvers and preconditioners avialable to an application regardless of 
the format of that data provided by those applications.  

R5.1: The user will choose which linear solver or preconditioner
to apply to the data. (A set of valid choices will be provided).

R5.1: This should be accomplished primarily by "translating" the 
application data into the storage schema required by the solver or
preconditioner or by applting the coefficient access method 
required by the solver or preconditioner.
\end{cxxdoc}
\end{cxxentry}
\begin{cxxentry}
{}
        {Non-Functional Requirements}
        {}
        {}
        {1.1.2}
\begin{cxxdoc}

R6.	Hypre should be portable across these platforms: Sun, DEC, BLue, and

R7.	Hypre should work with applications written in C, C++, FORTRAN77, and

R8.	Hypre should should be easy to compile, configure, and install.

R9.	Hypre should be scalable.

R10.	Hypre should work with applications that utilize the MPI standard
as well as those incorporating both MPI and OpenMP or MPI and Pthreads.

\end{cxxdoc}
\end{cxxentry}
\end{cxxentry}
\begin{cxxentry}
{}
        {Hypre Design}
        {}
        {}
        {1.2}
\begin{cxxnames}
\cxxitem{}
        {Software Architecture}
        {}
        {}
        {}
\label{cxx.1.2.1}
\cxxitem{}
        {Object Model}
        {}
        {}
        {}
\label{cxx.1.2.2}
\end{cxxnames}
\end{cxxentry}
\begin{cxxentry}
{}
        {Makefile Standards}
        {}
        {}
        {1.3}
\begin{cxxdoc}

Hypre is using Autoconf for installation.  (See Installation Procedures section).  This requires that a file called Makefile.in be created in each directory.  These files will generate Makefiles when the configure script is run at the top-level directory.  See example Makefile.in to see how it difeers from a conventional MAkefile.

\end{cxxdoc}
\end{cxxentry}
\begin{cxxentry}
{}
        {Documentation Standards}
        {}
        {}
        {1.4}
\begin{cxxdoc}
DOC++ is to be used for documentation.

Example documentation for C routines can be found in the directory
`struct_matrix_vector' in the files:

DevManual.dxx
communication.c

Initial user's and developer's manuals need to be constructed in
DOC++ to get things started.

The above `.dxx' file could serve as a beginning developer's manual.

\end{cxxdoc}
\end{cxxentry}
\begin{cxxentry}
{}
        {Coding Standards}
        {(...)}
        {}
        {1.5}
\begin{cxxdoc}
These guidelines are based on those provided by the ISE project in CASC.
See `file:/home/casc/software_development/html/index.html' for more info.

--------------------------------------------------------------------------

GENERAL GUIDELINES

The guidelines given in the ISE project document `coding_style.ps'
should be followed.  There are a few additions/changes that should
be made here:

1. The "order of parameters" section should be modified so that
opaque object names always appear first in the argument list.
For example, a routine for setting the coefficients of a matrix
object should look like

HYPRE_SetMatrixCoeffs(matrix_object, ...)

even though the matrix_object is an I/O parameter.  This is
just to provide some consistency.

2. Guidelines for indentation are given below.

3. Guidelines for documentation are given below.

--------------------------------------------------------------------------

NAMING CONVENTIONS FOR HYPRE


1. Functions and macros with parameters should be named as follows:

HYPRE_MixedCase(...)          /* User interface routines 
\end{cxxdoc}
\end{cxxentry}
\end{cxxentry}
\begin{cxxentry}
{}
        {Quaility Assurance Issues}
        {}
        {}
        {2}
\end{cxxentry}
\begin{cxxentry}
{}
        {Installation Procedures}
        {}
        {}
        {3}
\begin{cxxdoc}


Using Autoconfig in hypre
-------------------------

NOTE: These procedures currently only apply to the following directories:

utilities
\newline
struct_matrix_vector
\newline
struct_linear_solvers
\newline
test
\newline

In the top level linear_solvers directory type 'autoconf'.  This will use the file, configure.in, to create the file, configure.  This step only needs to be done when configure.in is changed.  

To automatically generate machine specific makefiles, type `configure'
in the `linear_solvers' directory.  The configure script is a portable
script generated by GNU Autoconf.  It runs a series of tests to
determine characteristics of the machine on which it is running, and
it uses the results of the these tests to produce the machine specific
makefiles, called `Makefile', from template files called `Makefile.in'
in each directory.  Once the makefiles are produced you can run make
as you would with any other makefile.

The configure script produces a file called `config.cache' which
stores some of its results.  If you wish to run configure again in a
way that will get different results, you should remove `config.cache'.

This configure script primarily does the following things:

- selects a C compiler
- provides either optimization or debugging options for the C compiler
- finds the headers and libraries for MPI

The configure script has some command-line options that can give you
some control over the choices it will make.  You can type

configure --help

to see the list of all of the command-line options to configure, but
the most significant options are at the bottom of the list, after the
line that reads "--enable and --with options recognized:"

--with-CC=ARG		
This option allows you to choose the C compiler
you wish to use.  The default compiler that
configure chooses is gcc, if it is available.

--enable-opt-debug=ARG  
Choose whether you want the C compiler to have
optimization or debugging flags.  For debugging,
replace `ARG' with `debug'.  For optimization,
replace `ARG' with `opt'.  If you want both sets
of flags, replace `ARG' with `both'.  The default
is optimization

--without-MPI		
This flag suppresses the use of MPI.  It produces
makefiles that serve the same purpose as the
files `Makefile.seq'

--with-mpi-include=DIR
--with-mpi-libs=LIBS
--with-mpi-lib-dirs=DIRS
These three flags are to be used if you want to
override the automatic search for MPI.  If you use
one of these flags, you must use all three.
Replace `DIR' with the path of the directory that
contains mpi.h, replace `LIBS' with a list of the
stub names of all the libraries needed for MPI, and
replace `DIRS' with a list of the directory paths
containing the libraries specified by `LIBS'
NOTE:  The lists `LIBS' and `DIRS' should be
space-separated and contained in quotes, e.g.
--with-mpi-libs="nsl socket mpi" and
--with-mpi-lib-dirs="/usr/lib /usr/local/mpi/lib"

--with-mpi-flags=FLAGS	
Sometimes other compiler flags are needed for
certain MPI implementations to work.  Replace
`FLAGS' with a space-separated list of whatever
flags are necessary.  This option does not
override the automatic search for MPI.  It can be
used to add to the results of the automatic search,
or it can be used along with the three previous
flags.

--with-MPICC=ARG	
The automatic search for the MPI stuff is based on
a tool such as `mpicc' that configure finds in
your $PATH.  If there is an implementation of MPI
that you wish to use, you can replace `ARG' with
the name of that MPI version's C compiler wrapper,
if it has one.  (MPICH has mpicc, IBM MPI has
mpcc, other MPIs use other names.)  configure will
then automatically find the necessary libraries
and headers.




Instructions for updating and maintaining installations of hypre

----------------------------------------------------------------------------

GENERAL COMMENTS:

There are two types of hypre installations that are updated and
maintained by members of the Scalable Linear Solvers project:

1. The "internal" installation:

- intended to give users access to relatively recent changes.
- updated somewhat frequently (about once a week).
- may be relatively unstable.
- one old installation is retained.
- installed on:
CASC Sun workstation cluster

2. The "external" installation:

- intended to give users consistent access across several platforms.
- updated less frequently than internal installation.
- should be relatively stable.
- a few old installations are retained.
- installed on:
CASC Sun workstation cluster
Compass DEC cluster
Blue-Pacific
Forest DEC cluster  (classified)
Blue(Sky)-Pacific   (classified)

Rob Falgout is maintaining two lists of users who want to receive
notification of installation updates: one for "internal" updates and
one for "external" updates.  Users should send email to Rob at
rfalgout@llnl.gov to get on either list.

----------------------------------------------------------------------------

INSTALLATION PROCEDURES:

All installations are done with the `update' Bourne-shell script
located in the top-level directory of the hypre repository.  To
get general usage info, do `update -help'.

Internal installation:

(on CASC cluster only)

1. Do `update internal' - this checks out the current version of the
repository, tags it with a date (file `VERSION_DATE'), compiles it,
temporarily installs it, and sets the permissions of the temporary
installation to appropriate values.  This also creates the tar file
`UPDATE.tar', currently only used in the external installation.

2. Do `update internal install' - this saves the previously
installed version, moves the temporary installation into its
correct location, and possibly deletes some older installations.

Note that step 1 is not actually needed, but is recommended in case
something doesn't work as planned (e.g., failed checkout, failed
compile, ...).  Also, step 1 provides the opportunity to test the
library in non-CASC codes such as ARES or ALE3D via the temporary
installation in the `UPDATE' subdirectory.

3. Send email to appropriate users indicating the update.

External installation:

(on CASC cluster)

1. Do `update external'
2. Do `update external install'

(on all other supported machines, e.g. the Compass cluster)

1. FTP the file `UPDATE.tar' to west (for example), and put it
in the install directory (do `update -dirs' for a list).
2. Do `tar xf UPDATE.tar'
3. Do `update external -dec '
4. Do `update external -dec install'

----------------------------------------------------------------------------

\end{cxxdoc}
\end{cxxentry}
\begin{cxxentry}
{}
        {Code Reference}
        {}
        {}
        {4}
\end{cxxentry}
\end{document}
